% Options for packages loaded elsewhere
\PassOptionsToPackage{unicode}{hyperref}
\PassOptionsToPackage{hyphens}{url}
%
\documentclass[
]{article}
\usepackage{amsmath,amssymb}
\usepackage{lmodern}
\usepackage{iftex}
\ifPDFTeX
  \usepackage[T1]{fontenc}
  \usepackage[utf8]{inputenc}
  \usepackage{textcomp} % provide euro and other symbols
\else % if luatex or xetex
  \usepackage{unicode-math}
  \defaultfontfeatures{Scale=MatchLowercase}
  \defaultfontfeatures[\rmfamily]{Ligatures=TeX,Scale=1}
\fi
% Use upquote if available, for straight quotes in verbatim environments
\IfFileExists{upquote.sty}{\usepackage{upquote}}{}
\IfFileExists{microtype.sty}{% use microtype if available
  \usepackage[]{microtype}
  \UseMicrotypeSet[protrusion]{basicmath} % disable protrusion for tt fonts
}{}
\makeatletter
\@ifundefined{KOMAClassName}{% if non-KOMA class
  \IfFileExists{parskip.sty}{%
    \usepackage{parskip}
  }{% else
    \setlength{\parindent}{0pt}
    \setlength{\parskip}{6pt plus 2pt minus 1pt}}
}{% if KOMA class
  \KOMAoptions{parskip=half}}
\makeatother
\usepackage{xcolor}
\usepackage[margin=1in]{geometry}
\usepackage{color}
\usepackage{fancyvrb}
\newcommand{\VerbBar}{|}
\newcommand{\VERB}{\Verb[commandchars=\\\{\}]}
\DefineVerbatimEnvironment{Highlighting}{Verbatim}{commandchars=\\\{\}}
% Add ',fontsize=\small' for more characters per line
\usepackage{framed}
\definecolor{shadecolor}{RGB}{248,248,248}
\newenvironment{Shaded}{\begin{snugshade}}{\end{snugshade}}
\newcommand{\AlertTok}[1]{\textcolor[rgb]{0.94,0.16,0.16}{#1}}
\newcommand{\AnnotationTok}[1]{\textcolor[rgb]{0.56,0.35,0.01}{\textbf{\textit{#1}}}}
\newcommand{\AttributeTok}[1]{\textcolor[rgb]{0.77,0.63,0.00}{#1}}
\newcommand{\BaseNTok}[1]{\textcolor[rgb]{0.00,0.00,0.81}{#1}}
\newcommand{\BuiltInTok}[1]{#1}
\newcommand{\CharTok}[1]{\textcolor[rgb]{0.31,0.60,0.02}{#1}}
\newcommand{\CommentTok}[1]{\textcolor[rgb]{0.56,0.35,0.01}{\textit{#1}}}
\newcommand{\CommentVarTok}[1]{\textcolor[rgb]{0.56,0.35,0.01}{\textbf{\textit{#1}}}}
\newcommand{\ConstantTok}[1]{\textcolor[rgb]{0.00,0.00,0.00}{#1}}
\newcommand{\ControlFlowTok}[1]{\textcolor[rgb]{0.13,0.29,0.53}{\textbf{#1}}}
\newcommand{\DataTypeTok}[1]{\textcolor[rgb]{0.13,0.29,0.53}{#1}}
\newcommand{\DecValTok}[1]{\textcolor[rgb]{0.00,0.00,0.81}{#1}}
\newcommand{\DocumentationTok}[1]{\textcolor[rgb]{0.56,0.35,0.01}{\textbf{\textit{#1}}}}
\newcommand{\ErrorTok}[1]{\textcolor[rgb]{0.64,0.00,0.00}{\textbf{#1}}}
\newcommand{\ExtensionTok}[1]{#1}
\newcommand{\FloatTok}[1]{\textcolor[rgb]{0.00,0.00,0.81}{#1}}
\newcommand{\FunctionTok}[1]{\textcolor[rgb]{0.00,0.00,0.00}{#1}}
\newcommand{\ImportTok}[1]{#1}
\newcommand{\InformationTok}[1]{\textcolor[rgb]{0.56,0.35,0.01}{\textbf{\textit{#1}}}}
\newcommand{\KeywordTok}[1]{\textcolor[rgb]{0.13,0.29,0.53}{\textbf{#1}}}
\newcommand{\NormalTok}[1]{#1}
\newcommand{\OperatorTok}[1]{\textcolor[rgb]{0.81,0.36,0.00}{\textbf{#1}}}
\newcommand{\OtherTok}[1]{\textcolor[rgb]{0.56,0.35,0.01}{#1}}
\newcommand{\PreprocessorTok}[1]{\textcolor[rgb]{0.56,0.35,0.01}{\textit{#1}}}
\newcommand{\RegionMarkerTok}[1]{#1}
\newcommand{\SpecialCharTok}[1]{\textcolor[rgb]{0.00,0.00,0.00}{#1}}
\newcommand{\SpecialStringTok}[1]{\textcolor[rgb]{0.31,0.60,0.02}{#1}}
\newcommand{\StringTok}[1]{\textcolor[rgb]{0.31,0.60,0.02}{#1}}
\newcommand{\VariableTok}[1]{\textcolor[rgb]{0.00,0.00,0.00}{#1}}
\newcommand{\VerbatimStringTok}[1]{\textcolor[rgb]{0.31,0.60,0.02}{#1}}
\newcommand{\WarningTok}[1]{\textcolor[rgb]{0.56,0.35,0.01}{\textbf{\textit{#1}}}}
\usepackage{graphicx}
\makeatletter
\def\maxwidth{\ifdim\Gin@nat@width>\linewidth\linewidth\else\Gin@nat@width\fi}
\def\maxheight{\ifdim\Gin@nat@height>\textheight\textheight\else\Gin@nat@height\fi}
\makeatother
% Scale images if necessary, so that they will not overflow the page
% margins by default, and it is still possible to overwrite the defaults
% using explicit options in \includegraphics[width, height, ...]{}
\setkeys{Gin}{width=\maxwidth,height=\maxheight,keepaspectratio}
% Set default figure placement to htbp
\makeatletter
\def\fps@figure{htbp}
\makeatother
\setlength{\emergencystretch}{3em} % prevent overfull lines
\providecommand{\tightlist}{%
  \setlength{\itemsep}{0pt}\setlength{\parskip}{0pt}}
\setcounter{secnumdepth}{-\maxdimen} % remove section numbering
\ifLuaTeX
  \usepackage{selnolig}  % disable illegal ligatures
\fi
\IfFileExists{bookmark.sty}{\usepackage{bookmark}}{\usepackage{hyperref}}
\IfFileExists{xurl.sty}{\usepackage{xurl}}{} % add URL line breaks if available
\urlstyle{same} % disable monospaced font for URLs
\hypersetup{
  pdftitle={Crop Statistics 1900-2017 dataset analasys},
  pdfauthor={Jetze Luyten, Axel Van Gestel, David Silva Troya},
  hidelinks,
  pdfcreator={LaTeX via pandoc}}

\title{Crop Statistics 1900-2017 dataset analasys}
\author{Jetze Luyten, Axel Van Gestel, David Silva Troya}
\date{}

\begin{document}
\maketitle

\hypertarget{setup-the-enviroment}{%
\subsection{Setup the enviroment}\label{setup-the-enviroment}}

The first thing to setup for the analysts is the environment with the
required packages and settings.

\hypertarget{install-required-packages-and-load-required-libraries}{%
\subsubsection{Install required packages and load required
libraries}\label{install-required-packages-and-load-required-libraries}}

During this analysis we used the \texttt{tidyverse} package for reading,
cleaning and plotting the data and the \texttt{ggcorrplot} package to
visualize the correlation matrix into a heat map.

\begin{Shaded}
\begin{Highlighting}[]
\CommentTok{\# install.packages("tidyverse")}
\CommentTok{\# install.packages("ggcorrplot")}
\FunctionTok{library}\NormalTok{(tidyverse) }\CommentTok{\# Contains all tidyverse packages (ggplot2, dplyr, ...)}
\FunctionTok{library}\NormalTok{(readxl) }\CommentTok{\#  Need to load explicitly (not a core tidyverse package)}
\FunctionTok{library}\NormalTok{(ggcorrplot) }\CommentTok{\# Used for generating correlation heatmaps (uses ggplot2)}
\FunctionTok{library}\NormalTok{(}\StringTok{"scales"}\NormalTok{)}
\end{Highlighting}
\end{Shaded}

\hypertarget{setup-enviroment-settings}{%
\subsubsection{Setup enviroment
settings}\label{setup-enviroment-settings}}

In the following code block we set the language R uses for it's messages
to English, clear all the global variables so that we always start with
a clean slate and setup ggplot to center the plot titles by default.

\begin{Shaded}
\begin{Highlighting}[]
\FunctionTok{Sys.setenv}\NormalTok{(}\AttributeTok{LANG =} \StringTok{"en"}\NormalTok{) }\CommentTok{\# Set language to English}
\FunctionTok{rm}\NormalTok{(}\AttributeTok{list =} \FunctionTok{ls}\NormalTok{()) }\CommentTok{\# Clears the Global Env}
\FunctionTok{theme\_update}\NormalTok{(}\AttributeTok{plot.title =} \FunctionTok{element\_text}\NormalTok{(}\AttributeTok{hjust =} \FloatTok{0.5}\NormalTok{)) }\CommentTok{\# Center all plot titles}
\end{Highlighting}
\end{Shaded}

\hypertarget{read-and-import-the-data-set}{%
\subsection{Read and import the data
set}\label{read-and-import-the-data-set}}

\hypertarget{read-the-data-set-uses-readr}{%
\subsubsection{Read the data set (uses
readr)}\label{read-the-data-set-uses-readr}}

\begin{Shaded}
\begin{Highlighting}[]
\NormalTok{column\_types }\OtherTok{\textless{}{-}} \FunctionTok{c}\NormalTok{(}
  \StringTok{"numeric"}\NormalTok{, }\CommentTok{\# ...1}
  \StringTok{"numeric"}\NormalTok{, }\CommentTok{\# Harvest\_year}
  \StringTok{"text"}\NormalTok{,    }\CommentTok{\# admin0}
  \StringTok{"text"}\NormalTok{,    }\CommentTok{\# admin1}
  \StringTok{"text"}\NormalTok{,    }\CommentTok{\# crop}
  \StringTok{"numeric"}\NormalTok{, }\CommentTok{\# hectares (ha)}
  \StringTok{"numeric"}\NormalTok{, }\CommentTok{\# production (tonnes)}
  \StringTok{"numeric"}\NormalTok{, }\CommentTok{\# year}
  \StringTok{"numeric"}\NormalTok{, }\CommentTok{\# yield(tonnes/ha)}
  \StringTok{"text"}\NormalTok{,    }\CommentTok{\# admin2}
  \StringTok{"text"}     \CommentTok{\# notes}
\NormalTok{)}
\NormalTok{crops }\OtherTok{\textless{}{-}} \FunctionTok{read\_xlsx}\NormalTok{(}
  \AttributeTok{path =} \StringTok{"./crops/food{-}twentieth{-}century{-}crop{-}statistics{-}1900{-}2017{-}xlsx.xlsx"}\NormalTok{,}
  \AttributeTok{sheet =} \StringTok{"CropStats"}\NormalTok{,}
  \AttributeTok{col\_types =}\NormalTok{ column\_types)}
\end{Highlighting}
\end{Shaded}

\begin{verbatim}
## New names:
## * `` -> `...1`
\end{verbatim}

\hypertarget{drop-not-needed-columns}{%
\subsection{Drop not needed columns}\label{drop-not-needed-columns}}

\begin{Shaded}
\begin{Highlighting}[]
\NormalTok{crops }\OtherTok{\textless{}{-}} \FunctionTok{select}\NormalTok{(crops, }\SpecialCharTok{{-}}\FunctionTok{c}\NormalTok{(...}\DecValTok{1}\NormalTok{, admin2, notes, Harvest\_year))}

\NormalTok{crops }\OtherTok{\textless{}{-}}\NormalTok{ crops }\SpecialCharTok{\%\textgreater{}\%} \FunctionTok{mutate}\NormalTok{(}\AttributeTok{crop =} \FunctionTok{factor}\NormalTok{(crop,}
                                        \AttributeTok{levels =} \FunctionTok{c}\NormalTok{(}\StringTok{"wheat"}\NormalTok{, }\StringTok{"winter wheat"}\NormalTok{,}
                                                   \StringTok{"spring wheat"}\NormalTok{, }\StringTok{"maize"}\NormalTok{,}
                                                   \StringTok{"cereals"}\NormalTok{),}
                                        \AttributeTok{ordered =} \ConstantTok{TRUE}\NormalTok{))}

\NormalTok{crops }\OtherTok{\textless{}{-}}\NormalTok{ crops }\SpecialCharTok{\%\textgreater{}\%} \FunctionTok{mutate}\NormalTok{(}\AttributeTok{year =} \FunctionTok{as.integer}\NormalTok{(year))}
\end{Highlighting}
\end{Shaded}

\hypertarget{clear-not-needed-variables}{%
\subsection{Clear not needed
variables}\label{clear-not-needed-variables}}

\begin{Shaded}
\begin{Highlighting}[]
\FunctionTok{rm}\NormalTok{(column\_types)}
\end{Highlighting}
\end{Shaded}

\hypertarget{filtering-and-cleaning}{%
\section{Filtering and cleaning}\label{filtering-and-cleaning}}

\hypertarget{check-for-the-number-of-nas-in-each-column}{%
\subsubsection{Check for the number of NA's in each
column}\label{check-for-the-number-of-nas-in-each-column}}

\begin{Shaded}
\begin{Highlighting}[]
\NormalTok{sanity\_check }\OtherTok{\textless{}{-}} \ControlFlowTok{function}\NormalTok{(my\_df) \{}
  \ControlFlowTok{for}\NormalTok{ (j }\ControlFlowTok{in} \DecValTok{1}\SpecialCharTok{:}\FunctionTok{ncol}\NormalTok{(my\_df)) \{}
    \FunctionTok{print}\NormalTok{(}\FunctionTok{paste}\NormalTok{(}\FunctionTok{names}\NormalTok{(my\_df[j]), }\StringTok{":"}\NormalTok{, }\FunctionTok{sum}\NormalTok{(}\FunctionTok{is.na}\NormalTok{(my\_df[, j]))))}
\NormalTok{  \}}
\NormalTok{\}}

\FunctionTok{sanity\_check}\NormalTok{(crops)}
\end{Highlighting}
\end{Shaded}

\begin{verbatim}
## [1] "admin0 : 0"
## [1] "admin1 : 2934"
## [1] "crop : 0"
## [1] "hectares (ha) : 1623"
## [1] "production (tonnes) : 1998"
## [1] "year : 0"
## [1] "yield(tonnes/ha) : 2013"
\end{verbatim}

\hypertarget{view-crop-tibble}{%
\subsubsection{View `crop' tibble}\label{view-crop-tibble}}

\begin{Shaded}
\begin{Highlighting}[]
\NormalTok{crops}
\end{Highlighting}
\end{Shaded}

\begin{verbatim}
## # A tibble: 36,707 x 7
##    admin0  admin1 crop  `hectares (ha)` `production (tonnes)`  year yield(tonn~1
##    <chr>   <chr>  <ord>           <dbl>                 <dbl> <int>        <dbl>
##  1 Austria <NA>   wheat              NA                    NA  1902         1.31
##  2 Austria <NA>   wheat              NA                    NA  1903         1.47
##  3 Austria <NA>   wheat              NA                    NA  1904         1.27
##  4 Austria <NA>   wheat              NA                    NA  1905         1.33
##  5 Austria <NA>   wheat              NA                    NA  1906         1.28
##  6 Austria <NA>   wheat              NA                    NA  1907         1.37
##  7 Austria <NA>   wheat              NA                    NA  1908         1.36
##  8 Austria <NA>   wheat              NA                    NA  1909         1.35
##  9 Austria <NA>   wheat              NA                    NA  1910         1.18
## 10 Austria <NA>   wheat              NA                    NA  1911         1.37
## # ... with 36,697 more rows, and abbreviated variable name
## #   1: `yield(tonnes/ha)`
\end{verbatim}

\hypertarget{correlation-heatmap-uses-ggcorrplot}{%
\subsection{Correlation heatmap (uses
ggcorrplot)}\label{correlation-heatmap-uses-ggcorrplot}}

\hypertarget{generate-a-correlation-heatmap-of-the-numeric-values}{%
\subsubsection{Generate a correlation heatmap of the numeric
values}\label{generate-a-correlation-heatmap-of-the-numeric-values}}

\begin{Shaded}
\begin{Highlighting}[]
\NormalTok{crops\_numeric }\OtherTok{\textless{}{-}} \FunctionTok{select}\NormalTok{(crops,}
                        \StringTok{\textasciigrave{}}\AttributeTok{hectares (ha)}\StringTok{\textasciigrave{}}\NormalTok{,}
                        \StringTok{\textasciigrave{}}\AttributeTok{production (tonnes)}\StringTok{\textasciigrave{}}\NormalTok{,}
\NormalTok{                        year ,}
                        \StringTok{\textasciigrave{}}\AttributeTok{yield(tonnes/ha)}\StringTok{\textasciigrave{}}\NormalTok{)}

\NormalTok{crops\_numeric\_corr }\OtherTok{\textless{}{-}} \FunctionTok{cor}\NormalTok{(crops\_numeric, }\AttributeTok{use =} \StringTok{"complete.obs"}\NormalTok{) }\CommentTok{\# Use only non NA}

\NormalTok{ggcorrplot}\SpecialCharTok{::}\FunctionTok{ggcorrplot}\NormalTok{(crops\_numeric\_corr,}
                       \AttributeTok{lab =} \ConstantTok{TRUE}\NormalTok{, }\CommentTok{\# Show correlation coefficients}
                       \AttributeTok{colors =} \FunctionTok{c}\NormalTok{(}\StringTok{"darkturquoise"}\NormalTok{, }\StringTok{"white"}\NormalTok{, }\StringTok{"salmon"}\NormalTok{),}
                       \AttributeTok{title =} \StringTok{"Correlation between the numeric values"}\NormalTok{)}
\end{Highlighting}
\end{Shaded}

\includegraphics{crops_files/figure-latex/unnamed-chunk-8-1.pdf}

\begin{Shaded}
\begin{Highlighting}[]
\CommentTok{\# Clear not needed variables}
\FunctionTok{rm}\NormalTok{(crops\_numeric, crops\_numeric\_corr)}
\end{Highlighting}
\end{Shaded}

\hypertarget{plots-and-stuff-uses-ggplot2}{%
\subsection{Plots and stuff (uses
ggplot2)}\label{plots-and-stuff-uses-ggplot2}}

\hypertarget{production-x-years}{%
\subsubsection{Production x years}\label{production-x-years}}

\begin{Shaded}
\begin{Highlighting}[]
\FunctionTok{ggplot}\NormalTok{(}\AttributeTok{data =}\NormalTok{ crops, }\AttributeTok{mapping =} \FunctionTok{aes}\NormalTok{(}\AttributeTok{x =}\NormalTok{ year, }\AttributeTok{y =} \StringTok{\textasciigrave{}}\AttributeTok{production (tonnes)}\StringTok{\textasciigrave{}}\NormalTok{)) }\SpecialCharTok{+}
  \FunctionTok{geom\_point}\NormalTok{(}\AttributeTok{alpha =} \DecValTok{1}\SpecialCharTok{/}\DecValTok{10}\NormalTok{) }\SpecialCharTok{+}
  \FunctionTok{stat\_smooth}\NormalTok{(}\AttributeTok{method =} \StringTok{"lm"}\NormalTok{, }\AttributeTok{formula =}\NormalTok{ y }\SpecialCharTok{\textasciitilde{}}\NormalTok{ x }\SpecialCharTok{+} \FunctionTok{I}\NormalTok{(x}\SpecialCharTok{\^{}}\DecValTok{2}\NormalTok{), }\AttributeTok{size =} \DecValTok{1}\NormalTok{)}\SpecialCharTok{+}
  \FunctionTok{ggtitle}\NormalTok{(}\StringTok{"Production over the years"}\NormalTok{)}
\end{Highlighting}
\end{Shaded}

\includegraphics{crops_files/figure-latex/unnamed-chunk-9-1.pdf}

\hypertarget{yield-outliers}{%
\subsubsection{Yield outliers}\label{yield-outliers}}

\begin{Shaded}
\begin{Highlighting}[]
\FunctionTok{ggplot}\NormalTok{(}\AttributeTok{data =}\NormalTok{ crops, }\AttributeTok{mapping =} \FunctionTok{aes}\NormalTok{(}\AttributeTok{x =}\NormalTok{ crop, }\AttributeTok{y =} \StringTok{\textasciigrave{}}\AttributeTok{yield(tonnes/ha)}\StringTok{\textasciigrave{}}\NormalTok{)) }\SpecialCharTok{+}
  \FunctionTok{geom\_boxplot}\NormalTok{() }\SpecialCharTok{+}
  \FunctionTok{ggtitle}\NormalTok{(}\StringTok{"Yield outliers"}\NormalTok{)}
\end{Highlighting}
\end{Shaded}

\includegraphics{crops_files/figure-latex/unnamed-chunk-10-1.pdf}

\hypertarget{count-of-crops}{%
\subsubsection{Count of crops}\label{count-of-crops}}

\begin{Shaded}
\begin{Highlighting}[]
\FunctionTok{ggplot}\NormalTok{(}\AttributeTok{data =}\NormalTok{ crops, }\AttributeTok{mapping =} \FunctionTok{aes}\NormalTok{(}\AttributeTok{x =}\NormalTok{ crop, }\AttributeTok{y =} \StringTok{\textasciigrave{}}\AttributeTok{yield(tonnes/ha)}\StringTok{\textasciigrave{}}\NormalTok{)) }\SpecialCharTok{+}
  \FunctionTok{geom\_boxplot}\NormalTok{() }\SpecialCharTok{+}
  \FunctionTok{ggtitle}\NormalTok{(}\StringTok{"Yield outliers"}\NormalTok{)}
\end{Highlighting}
\end{Shaded}

\includegraphics{crops_files/figure-latex/unnamed-chunk-11-1.pdf}

\hypertarget{section}{%
\subsubsection{?}\label{section}}

\begin{Shaded}
\begin{Highlighting}[]
\FunctionTok{ggplot}\NormalTok{(}\AttributeTok{data =}\NormalTok{ crops, }\AttributeTok{mapping =} \FunctionTok{aes}\NormalTok{(}\AttributeTok{x =}\NormalTok{ crop, }\AttributeTok{y =} \StringTok{\textasciigrave{}}\AttributeTok{hectares (ha)}\StringTok{\textasciigrave{}}\NormalTok{)) }\SpecialCharTok{+}
  \FunctionTok{geom\_boxplot}\NormalTok{() }\SpecialCharTok{+}
  \FunctionTok{scale\_y\_continuous}\NormalTok{(}\AttributeTok{labels =}\NormalTok{ comma) }\SpecialCharTok{+}
  \FunctionTok{ggtitle}\NormalTok{(}\StringTok{"Hectares outliers"}\NormalTok{)}
\end{Highlighting}
\end{Shaded}

\includegraphics{crops_files/figure-latex/unnamed-chunk-12-1.pdf}

\hypertarget{references}{%
\subsection{References}\label{references}}

\begin{itemize}
\tightlist
\item
  \href{http://www.sthda.com/english/wiki/ggcorrplot-visualization-of-a-correlation-matrix-using-ggplot2}{Correlation
  heatmap using ggplot2}
\end{itemize}

\end{document}
